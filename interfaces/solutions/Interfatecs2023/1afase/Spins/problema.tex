Um cientista chamado Doc. Hugo Strange fez diversos experimentos com \textit{spins} que são elementos capazes de trocar o status de \textit{quantum gates} quando passam por eles, os \textit{quantum gates} podem assumir o estados \textit{OPEN} e/ou \textit{CLOSED}, inclusive nos experimentos o Doc. Hugo Strange percebeu que toda a vez que um \textit{spin} é criado, um \textit{quantum gate} é criado, e tal spin é responsável por alterar seu estado. Entretanto, Doc. Hugo tem dificuldade em
determinar quais \textit{quantum gates} estarão abertos
após a criação de $N$ \textit{quantum gates}, uma vez que ele percebeu que os \textit{spins} só trocam o estado de \textit{quantum gates} múltiplos, por exemplo, o 3o. spin criado abrirá os \textit{quantum gates} 3, 6, 9, e assim por diante.

Considerando que a cada experimento $N$ \textit{spins} são criados e, consequentemente, $N$ \textit{quantum gates} estarão fechados, o Doc. Hugo Streange deseja saber quantos ficarão abertos após a liberação dos \textit{spins}, sendo que não é possível saber a ordem de liberação dos mesmos.

Entrada
A entrada contém vários casos de teste, onde cada caso existe um número inteiro $N$ $(0 < N < 45\times 10^{6})$ indicando a quantidade de \textit{spins} e \textit{quantum gates}, o fim da entrada é dada por $N=0$.

Saída
Para cada caso de teste deve-se produzir uma linha de saída indicando quais \textit{quantum gates}, em ordem crescente, estarão abertos após a liberação dos \textit{spins}. Você deve separar a sequência de \textit{_quantum gates_} por um espaço em branco.

2                        1
3                        1
4                        1 4
0
